\section{Implementation}
\label{sec:implementation}
% ALL

Ontoqa has been realized as a Java\footnote{Oracle Java SE 8} web and standalone application, packaged with Maven.
%
Ontoqa can be executed both as a standalone application and as a web application.
%
All functionalities has been tested carefully against 224 total unit tests.

Ontoqa leverages some well known technologies. Here we present them, giving an idea about how they have been used in our implementation. The reader may refer to the open source code of the project and the corresponding Javadocs to get into the implementation details.


\begin{itemize}
	\item[Ontology] INSERT HERE
	
	\item[Lexicon] INSERT HERE	
	
	\item[I/O] we used the Jackson core library and data-binding modules to implement data representation in JSON and YAML format.
	
	\item[CLI] we used Apache CLI to implement options and argument parsing for the command line interface.
	
	\item[Web UI] INSERT HERE
	
	\item[Web Service] we used Spring Framework and Spring MVC to implement the REST service interface.
	
	\item[Logging] we used SLF4J to implement logging ayer as facade, and Logback as the underlying logging framework.
	
	\item[Development] we used Lombok to implement bean constructors, getter/setters, thus minimizing code redundancy.
	
	\item[Testing] we used JUnit4 to implement unit testing suites.
\end{itemize}