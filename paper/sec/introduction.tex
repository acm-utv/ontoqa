\section{Introduction}
\label{sec:introduction}

% CONTEXT: NLP and QA
Natural Language Processing (NLP) is a theoretically motivated range of
computational techniques for analyzing and representing naturally occurring texts
at one or more levels of linguistic analysis for the purpose of achieving human-like
language processing for a range of tasks or applications \cite{liddy2001natural}.

NLP can disruptively reshape human-machine interaction, data-driven decision-making and daily exploration of knowledge by humans.
%
One of the most representative and widely studied NLP application is question-answering (Q\&A).
%
With the fast growing diffusion of semantic web and advancement in NLP technologies, Q\&A systems will be one of the most important interface to knowledge.

% CHALLENGE
It is well-known that it is not possible to develop a single ontology to effectively capture the whole knowledge.
%
This theoretical and intuitive limit makes necessary to develop Q\&A systems that can easily adapt to distinct ontologies and lexicons.

% GOAL
The goal of the presented research is to address this  challenge.
%
In this work we describe \textit{Ontoqa}, a Q\&A web and standalone application which aims to achieve this ambitious goal.
%
The proposed solution leverages (i) ontology-driven NLP through the use of the LTAG/DUDES model and (ii) a parsing algorithm that aims to reduce both the syntactic and  semantic search space \cite{cimiano2014ontology}.


% REMAINDER
The remainder of the paper is organized as follows:
Section~\ref{sec:background} gives the background context, useful to better understand our work;
Section~\ref{sec:architecture} shows the architecture implemented in our solution;
Section~\ref{sec:grammar} shows the grammar used to parse the natural language;
Section~\ref{sec:ontology} shows the ontology used to semantically represent the reference domain;
Section~\ref{sec:parsing} shows the algorithm used for question parsing;
Section~\ref{sec:implementation} shows how the application has been implemented, with a focus on the adopted technologies;
Section~\ref{sec:evaluation} shows the experiemntal results, focusing on both response time and memory usage;
Section~\ref{sec:sample-execution} shows some representative parsing executions on benchmark questions;
Section~\ref{sec:improvements} outlines the possible improvements for the proposed work;
Section~\ref{sec:conclusions} concludes this article, summarizing our work and results.