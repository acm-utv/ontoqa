\section{Ontology Lexica}
\label{sec:ontology-lexica}


The Ontology Lexica is generated by the lemon Design Pattern Library. We wrote a file with \textit{.ldp} extension containing the ontology lexica in lemon Design Pattern language and then we use the lemon patterns to convert it to \textit{RDF/XML} file. 
The semantic vocabulary used to represent the meaning of words have to correspond to the vocabulary of the ontology. So we matched the lemon model to the ontology in the \textit{.ldp} file as following:
\begin{itemize}
\item common nouns, e.g. \textit{company, person} and \textit{nation}, and their other forms (e.g. plural form) are \textit{ClassNouns} in lemon pattern language.
\item all the OWL properties, e.g. \textit{hasFounder, hasChairman, ...} are expressed as \textit{relational nouns} (\textit{RelationalNoun} in lemon pattern language) to represent relations between two entities that are expressed in sentences as nouns.
\item for named individuals in the domain we used the \textit{Name(...)} form.
\item for the verbs we needed in our domain we have defined the present and the past form but any other form can be easily added in the \textit{StateVerb(...)} pattern.
\item to solve the ambiguity we talked about in the \textit{Ontology} section we constructed adjectives, as \textit{italian, american} end so on,from values of the object property \textit{hasNation} (e.g \textit{Italy, United States,...} with the \textit{IntersectiveObjectPropertyAdjective(...)} pattern.
\item the superlative form is expressed as a \textit{ScalarAdjective} (in the \textit{covariant} form) in lemon language. Consider the \textit{valuable} adjective in the superlative form "\textit{the most valuable}", the extent to which a company is \textit{valuable} is proportional (or \textit{covariant}) to its market value.
\item Finally we defined relational adjectives as "\textit{headquartered in}" with the \textit{RelationalAdjective} form.
\end{itemize}
