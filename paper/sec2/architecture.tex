\section{Architecture}
\label{sec:architecture}
% ANTONELLA
The system created allows the user to enter a question in natural language, this question the system intends to translate it into a formal query through the ontology-based grammar-based language analysis and to which ontology answers in Based on the content of the application.
\begin{itemize}
	\item Grammar generation
	\item Linguistic analysis of demand and answer to the user
\end{itemize}

\subsection{Grammar generator}
Grammar has been assumed made up of two parts: it depends on ontology, independent of ontology. The domain specific grammar refers to the part that contains the lexical entries like individuals, concepts and properties contained in the ontology. The ontological independent part contains expressions like auxiliary verbs, determiners, wh-words, and so on.

Generally what he does is represented as follows:

\begin{figure}[H]
   \centering
    \includegraphics[scale=0.5]{./fig/grammar}
     \caption{}
    \label{fig: grammar}
\end{figure}

Both parts of the grammar use the main linguistic representations or each grammar is represented as a pair of syntactic and semantic representations. As a syntactic representation, Lexicalized Tree Adjoining Grammar (LTAG) is used. As semantic representations we take Dudes.

The first step in generating a grammar from a given ontology is to enrich the ontology with information about its verbalization. The framework we use for this is LexInfo, which otheers a general frame for creating a declarative specidication of the lexicon-ontology interface by connecting concepts of the ontology to information about their linguistic realization, i.e. word forms, morphology, sub-categoriziation frames and how syntactic and semantic arguments correspond to each other. The lexical entries specified by LexInfo are then input to a general mechanism for generating grammar entries, i.e. pairs of syntactic and semantic
representations.

\subsection{Ontoqa}
The system allows a user to submit, via an intuitive interface, a natural-language question in English.

Given the question in Natural Language, a parsing is applied that constructs an LTAG derivative tree considering only the syntactic part of the grammatical voices involved.

Subsequently, syntactic and semantic composition rules apply to the construction of a tree in accordance with the LTAG derivation tree and DUDES semantics.

In general, what happens is that LTAG provides substitution and adjunction rules while DUDES semantics provides rules as saturation operations that are interpreted as substitutions and a union operation that interprets the adjoin.

After the disambiguation and obtaining of the SLTAG of our universe of speech, we generate a formal query. The formal query will be submitted to the ontology that will return us an answer. The procedure is shown in the figure.

\begin{figure}[H]
   \centering
    \includegraphics[scale=0.5]{./fig/ontoqa}
     \caption{}
    \label{fig: ontoqa}
\end{figure}